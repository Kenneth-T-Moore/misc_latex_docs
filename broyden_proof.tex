\documentclass{article}

\usepackage[numbers]{natbib}

\usepackage{graphicx}

% use Times fonts if available on your TeX system
\usepackage{mathptmx}

% insert here the call for the packages your document requires
\usepackage{hyperref}
\usepackage{fixltx2e}
\hypersetup{
    colorlinks=true,
    linkcolor=blue,
    filecolor=blue,
    urlcolor=blue,
    citecolor=blue
}

\urlstyle{same}

\usepackage{bm}
\usepackage{gensymb}
\usepackage{amsmath}
\usepackage{amstext}
\usepackage{commath}

\usepackage{listings}
\usepackage[svgnames]{xcolor}
\lstset{language=Python,
  basicstyle=\small,
  keywordstyle=\color{DodgerBlue}\ttfamily,
  identifierstyle=\color{black}\ttfamily,
  commentstyle=\color{FireBrick}\bfseries\ttfamily,
  stringstyle=\color{OliveDrab}\ttfamily}

\usepackage{soul}
\usepackage{algorithm}
\usepackage[noend]{algpseudocode}

\soulregister\cite7
\soulregister\ref7
\soulregister\pageref7

\DeclareMathAlphabet{\mathcal}{OMS}{cmsy}{m}{n}
\newcommand\p{\partial}
\newcommand\Y{\mathcal{Y}}
\newcommand\R{\mathcal{R}}
\newcommand\F{\mathcal{F}}

\begin{document}

\section{Broyden's Second Method}

Broyden update to the approximation of the inverse Jacobian:

\begin{equation}
  J_n^{-1} = J_{n-1}^{-1} + \frac{\Delta x_n - J_{n-1}^{-1} \Delta r_n}{\| \Delta r_{n}^2 \|} \Delta r_{n}^T   \label{eq:broyden}
\end{equation}
Quasi-Newton step with updated Jacobian.

\begin{equation}
  \Delta x_n \cong J_n^{-1} \Delta r_n = J_n^{-1} (0 - r_n) = -J_n^{-1} r_n  \label{eq:newton_update}
\end{equation}

\section{Model Structure}

The model consists of some number of implicit relationships whose states are to be solved. We split this into two subproblems with
states $x$ and $y$ that satisfy the residual equations $r_1$ and $r_2$.

\begin{equation}
  r_1 = F(x, y)  \label{eq:outer_eq}
\end{equation}
\begin{equation}
  r_2 = G(x, y)  \label{eq:inner_eq}
\end{equation}
The model also has two solvers: an outer solver that drives the full solution, and an inner solver that reconverges each
iteration of the outer solver. The inner solver will always take residual equation $r_2$ and solve for $y$.

\section{Recursive Broyden}

In the recursive solution strategy, the inner solver converges $r_2$ and solves for $y$, while the outer solver converges
$r_1$ and solves for $x$.  It has no knowledge of $y$.

To apply Broyden's method to the outer system, we need to differentiate equation~\eqref{eq:outer_eq} with respect to $x$.

\begin{equation}
  dr_1 = \frac{\partial F}{\partial x} dx + \frac{\partial F}{\partial y} \frac{dy}{dx} dx  \label{eq:deriv_outer}
\end{equation}
To get the derivative of $y$ with respect to $x$, we need to also differentiate the inner solver. The underlying assumption is that the
inner solver is fully converged, so that the residual $r_2$ is zero.

\begin{equation}
  dr_2 = \frac{\partial G}{\partial x} dx + \frac{\partial G}{\partial y} \frac{dy}{dx} dy = 0
\end{equation}
\begin{equation}
  \frac{dy}{dx} = {\frac{\partial G}{\partial y}}^{-1} \frac{\partial G}{\partial x}
\end{equation}
Substituting back into equation~\eqref{eq:deriv_outer} gives us:

\begin{equation}
  dr_1 = \frac{\partial F}{\partial x} dx + \frac{\partial F}{\partial y} {\frac{\partial G}{\partial y}}^{-1} \frac{\partial G}{\partial x} dx
\end{equation}
\begin{equation}
  \frac{dy}{dr_1} = \left[ \frac{\partial F}{\partial x} + \frac{\partial F}{\partial y} {\frac{\partial G}{\partial y}}^{-1} \frac{\partial G}{\partial x} \right]^{-1}
\end{equation}
To clean things up, and since this term will appear again, we define:

\begin{equation}
  \beta = \left[ \frac{\partial F}{\partial x} + \frac{\partial F}{\partial y} {\frac{\partial G}{\partial y}}^{-1} \frac{\partial G}{\partial x} \right]^{-1} \label{eq:beta}
\end{equation}
Now we can substitute into equation~\eqref{eq:broyden} to calculate the Broyden update to the inverse Jacobian for this case, where the inverse Jacobian is just the
derivative of $y$ with respect to $r_1$.

\begin{equation}
  J_n^{-1} = \beta + \frac{\Delta x - \beta \Delta r_{1}}{\Delta r_{1}^2 } \Delta r_{1}^T
\end{equation}
Finally, if we plug this into into equation~\eqref{eq:newton_update}, we get the commanded update to the state value.

\begin{equation}
  \Delta x_n \cong  -(\beta + \frac{\Delta x - \beta \Delta r_{1}}{\Delta r_{1}^2 } \Delta r_{1}^T )r_{1} \label{eq:recursive_final}
\end{equation}


\section{Hierarchical Broyden}

In the hierarchical solution strategy, the inner solver converges $r_2$ and solves for $y$ as before, however the outer solver sees both residuals
$r_1$ and $r_2$ (which gets converged to zero by the subsolver). The outer solver also can set $y$, and while the inner solver may use that as an initial
guess, it finds its own solution for $y$.

To apply Broyden's method to the hierarchical problem, we need to differentiate both equation~\eqref{eq:outer_eq} and equation~\eqref{eq:inner_eq} with respect
to $x$ and $r$. The Jacobian is just:

\begin{equation}
  J_{n-1} =
  \begin{bmatrix}
     \frac{\partial F}{\partial x}  &  \frac{\partial F}{\partial y} \\
     \frac{\partial G}{\partial x}  &  \frac{\partial G}{\partial y}
  \end{bmatrix}
\end{equation}
We need the inverse Jacobian, so we use the closed form solution for the block inverse of a two by two matrix.

\begin{equation}
  J_{n-1}^{-1} =
  \begin{bmatrix}
    \left[ \frac{\partial F}{\partial x} + \frac{\partial F}{\partial y} {\frac{\partial G}{\partial y}}^{-1} \frac{\partial G}{\partial x} \right]^{-1} &
    - {\frac{\partial F}{\partial x}}^{-1} \frac{\partial F}{\partial y} \left[ \frac{\partial G}{\partial y} + \frac{\partial G}{\partial x} {\frac{\partial F}{\partial x}}^{-1} \frac{\partial F}{\partial y} \right]^{-1} \\
    - {\frac{\partial G}{\partial y}}^{-1} \frac{\partial G}{\partial x} \left[ \frac{\partial F}{\partial x} + \frac{\partial F}{\partial y} {\frac{\partial G}{\partial y}}^{-1} \frac{\partial G}{\partial x} \right]^{-1} &
    \left[ \frac{\partial G}{\partial y} + \frac{\partial G}{\partial x} {\frac{\partial F}{\partial x}}^{-1} \frac{\partial F}{\partial y} \right]^{-1}
  \end{bmatrix}
\end{equation}
We can simplify this a litle by substituting in equation~\eqref{eq:beta}

\begin{equation}
  J_{n-1}^{-1} =
  \begin{bmatrix}
    \beta &
    - {\frac{\partial F}{\partial x}}^{-1} \frac{\partial F}{\partial y} \left[ \frac{\partial G}{\partial y} + \frac{\partial G}{\partial x} {\frac{\partial F}{\partial x}}^{-1} \frac{\partial F}{\partial y} \right]^{-1} \\
    - {\frac{\partial G}{\partial y}}^{-1} \frac{\partial G}{\partial x} \beta &
    \left[ \frac{\partial G}{\partial y} + \frac{\partial G}{\partial x} {\frac{\partial F}{\partial x}}^{-1} \frac{\partial F}{\partial y} \right]^{-1}
  \end{bmatrix}
\end{equation}
We can define one more term to make this simpler:

\begin{equation}
  \gamma = \left[ \frac{\partial G}{\partial y} + \frac{\partial G}{\partial x} {\frac{\partial F}{\partial x}}^{-1} \frac{\partial F}{\partial y} \right]^{-1}
\end{equation}

\begin{equation}
  J_{n-1}^{-1}   =
  \begin{bmatrix}
    \beta &
    - {\frac{\partial F}{\partial x}}^{-1} \frac{\partial F}{\partial y} \gamma \\
    - {\frac{\partial G}{\partial y}}^{-1} \frac{\partial G}{\partial x} \beta &
    \gamma
  \end{bmatrix}
\end{equation}
We now have the inverse Jacobian we need for equation~\eqref{eq:broyden}, but now we need to expand the state and residual vectors.

\begin{equation}
  \Delta r_n = {
  \begin{bmatrix}
    \Delta r_1 \\
    0
  \end{bmatrix}
  }_n
\end{equation}
\begin{equation}
  \Delta r_n = {
  \begin{bmatrix}
    \Delta x \\
    \Delta y
  \end{bmatrix}
  }_n
\end{equation}
The residual for state $y$ as seen by the outer solver is zero because the inner solver converged the inner equation. Now, let's build up
equation~\eqref{eq:broyden} one step at a time to see each matrix multiplication.

\begin{equation}
  J_{n-1}^{-1} \Delta r_n = J_{n-1}^{-1} {
  \begin{bmatrix}
    \Delta r_1 \\
    0
  \end{bmatrix}
  }_n =
  \begin{bmatrix}
    \beta \Delta r_1 \\
    - {\frac{\partial G}{\partial y}}^{-1} \frac{\partial G}{\partial x} \beta \Delta r_1
  \end{bmatrix}
\end{equation}
Since the residual $r_2$ is zero:

\begin{equation}
  \| \Delta r_{n}^2 \| = \Delta r_{1}^2
\end{equation}
Putting it all together, we get:

\begin{equation}
  J_n^{-1} =
  \begin{bmatrix}
    \beta &
    - {\frac{\partial F}{\partial x}}^{-1} \frac{\partial F}{\partial y} \gamma \\
    - {\frac{\partial G}{\partial y}}^{-1} \frac{\partial G}{\partial x} \beta &
    \gamma
  \end{bmatrix}
  +
  \frac{
  \begin{bmatrix}
    \Delta x \\
    \Delta y
  \end{bmatrix}
   -
  \begin{bmatrix}
    \beta \Delta r_1 \\
    - {\frac{\partial G}{\partial y}}^{-1} \frac{\partial G}{\partial x} \beta \Delta r_1
  \end{bmatrix}
  }{\Delta r_{1}^2}
  \begin{bmatrix}
    \Delta r_{1}^T &
    0
  \end{bmatrix}
\end{equation}
Performing the outer product, and reducing gives:

\begin{equation}
  \Delta J_n^{-1} =
  \begin{bmatrix}
    \beta + \frac{\Delta x - \beta \Delta r_{1}}{\Delta r_{1}^2 } \Delta r_{1}^T &
    {\frac{\partial F}{\partial x}}^{-1} \frac{\partial F}{\partial y} \gamma \\
    {\frac{\partial G}{\partial y}}^{-1} \frac{\partial G}{\partial x} \beta \frac{\Delta x - {\frac{\partial G}{\partial y}}^{-1} \frac{\partial G}{\partial x} \beta \Delta r_{1}}{\Delta r_{1}^2 } \Delta r_{1}^T  &
    \gamma
  \end{bmatrix}
\end{equation}
Now we plug this into into equation~\eqref{eq:newton_update}, to get the commanded update to the state value.

\begin{equation}
  {
  \begin{bmatrix}
    \Delta x \\
    \Delta y
  \end{bmatrix}
  }_n \cong -
  \begin{bmatrix}
    \beta + \frac{\Delta x - \beta \Delta r_{1}}{\Delta r_{1}^2 } \Delta r_{1}^T &
    {\frac{\partial F}{\partial x}}^{-1} \frac{\partial F}{\partial y} \gamma \\
    {\frac{\partial G}{\partial y}}^{-1} \frac{\partial G}{\partial x} \beta + \frac{\Delta y - {\frac{\partial G}{\partial y}}^{-1} \frac{\partial G}{\partial x} \beta \Delta r_{1}}{\Delta r_{1}^2 } \Delta r_{1}^T  &
    \gamma
  \end{bmatrix}
  {
  \begin{bmatrix}
    r_1 \\
    0
  \end{bmatrix}
  }_n
\end{equation}

\begin{equation}
  {
  \begin{bmatrix}
    \Delta x \\
    \Delta y
  \end{bmatrix}
  }_n \cong -
  \begin{bmatrix}
    (\beta + \frac{\Delta x - \beta \Delta r_{1}}{\Delta r_{1}^2 } \Delta r_{1}^T) r_1 \\
    ({\frac{\partial G}{\partial y}}^{-1} \frac{\partial G}{\partial x} \beta + \frac{\Delta y - {\frac{\partial G}{\partial y}}^{-1} \frac{\partial G}{\partial x} \beta \Delta r_{1}}{\Delta r_{1}^2 } \Delta r_{1}^T) r_1
  \end{bmatrix}
\end{equation}
Finally, we can pull out the Broyden updated $\Delta x$:

\begin{equation}
  \Delta x_n \cong  -(\beta + \frac{\Delta x - \beta \Delta r_{1}}{\Delta r_{1}^2 } \Delta r_{1}^T )r_{1} \label{eq:hier_final}
\end{equation}
This matches the result from Recursive Broyden in equation~\eqref{eq:recursive_final} We also end up with an update to $y$, but the subsolver overrides this by solving that
subsystem.

\end{document}